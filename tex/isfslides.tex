\usepackage[english]{babel}
%\usepackage[utf8]{inputenc}
\usepackage{graphics}

\usetheme[%
%nexus,%        Nexus Fonts benutzen
%lnum,%         Versalziffern verwenden
%cmyk,%<rgbprint>,          Auswahl des Farbmodells
blue,%<blue/orange/green/violet> Auswahl des Sekund�rfarbklangs
medium,%<light,medium,dark>        Auswahl der Helligkeit
colorhead,%    Farbig hinterlegte Kopfleiste
colorfoot,%    Farbig hinterlegt Fu�leiste auf Titelseite
colorblocks,%   Bl�cke Farbig hinterlegen
%nopagenum,%    Keine Seitennumer in Fu�zeile
nodate,%       Kein Datum in Fu�leiste
%tocinheader,%   Inhaltsverzeichnis in Kopfleiste
%tinytocinheader,% kleines Kopfleisten-Inhaltsverzeichnis
%widetoc,%      breites Kopfleisten-Inhaltsverzeichnis
%narrowtoc,%    schmales Kopfleisten-Inhaltsverzeichnis
%nosubsectionsinheader,%  Keine subsections im Kopfleisten-Inhaltsverzeichnis
%nologoinfoot,% Kein Logo im Fu�bereich darstellen
]{tubs}

%\titlegraphic{\tuDefaultTitlegraphic}
\titlegraphic[scaled]{\includegraphics{../../../pics/logos/tubs-iz}}
\logo{\includegraphics{../../../pics/logos/isf}}

% my commands

\newcommand{\ifprint}[2]{#2}
\newcommand{\currentcolor}{blue}

%\usepackage[utf8]{inputenc} % Umlaute, other option latin1
%\usepackage{amsmath}
%\usepackage{amssymb}
\usepackage{booktabs} % toprule, midrule, bottomrule
\usepackage{adjustbox} % frame in includegraphics [export]

% tables
\usepackage{array} % tabular m{1cm}
\usepackage{ragged2e} % own column types P{1cm}
\newcolumntype{P}[1]{>{\RaggedRight\hspace{0pt}}p{#1}}
\newcolumntype{M}[1]{>{\RaggedRight\hspace{0pt}}m{#1}}

% TODOs
\newcommand{\todo}[1]{{\color{red} [#1]}}
\newcommand{\todoo}[1]{{\color{blue} \{#1\}}}
\newcommand{\todooo}[1]{{\color{green} (#1)}}
\newcommand{\changed}[1]{{\color{blue} #1}}
\newcommand{\changedd}[1]{{\color{blue} #1}}
\newcommand{\changeddd}[1]{{\color{blue} #1}}
%\newcommand{\changed}[1]{{\textbf{#1}}}
\newcommand{\fodo}[1]{\todo{\footnote{\todo{#1}}}}
\newcommand{\fodoo}[1]{\todoo{\footnote{\todoo{#1}}}}
\newcommand{\fodooo}[1]{\todooo{\footnote{\todooo{#1}}}}
\newcommand{\tocite}[1]{{\color{red} [cite:#1]}}
\newcommand{\todots}{\todo{\ldots}}
\newcommand{\kword}[1]{{\LARGE #1}}

%\renewcommand{\todo}[1]{}
%\renewcommand{\todots}{}
%\renewcommand{\tocite}[1]{}

%\renewcommand{\todoo}[1]{}
%\renewcommand{\todooo}[1]{}
%\renewcommand{\changed}[1]{#1}
%\renewcommand{\changedd}[1]{#1}
%\renewcommand{\changeddd}[1]{#1}
%\renewcommand{\fodo}[1]{}
%\renewcommand{\fodoo}[1]{}
%\renewcommand{\kword}[1]{#1}

\newcommand{\feature}[1]{\emph{#1}}
\newcommand{\spl}[1]{\emph{#1}}
\newcommand{\code}[1]{\mbox{\sf#1}}
\newcommand{\tool}[1]{\textsc{#1}}

\newcommand{\KeY}{Ke\kern-0.1emY}
\newcommand{\MonKeY}{Mon\KeY}
%\newcommand{\JPF}{\tool{Java PathFinder}}
\newcommand{\JPF}{\tool{JPF}}

% Email-Adressen
%\newcommand{\email}[1]{
	%\href{mailto:#1}{#1}
%}

% Aussagenlogische Ausdr�cke
\newcommand{\pand}{\wedge}
\newcommand{\por}{\vee}
\newcommand{\pnot}{\neg}
\newcommand{\pequals}{\Leftrightarrow}
\newcommand{\pimplies}{\Rightarrow}
\newcommand{\pnimplies}{\nRightarrow}
\newcommand{\patmostone}{\mbox{\textit{atmost1}}}
\newcommand{\pchooseone}{\mbox{\textit{choose1}}}
\newcommand{\ptrue}{\mbox{\textit{true}}}
\newcommand{\pfalse}{\mbox{\textit{false}}}
\newcommand{\true}{\texttt{true}}
\newcommand{\false}{\texttt{false}}

% Richtig, falsch
\newcommand{\richtig}{
	\begin{tikzpicture}[yscale=.2,xscale=.2,overlay]
		\draw[line width=4,line join=bevel,green] (0,1) to [bend left=30] (1,0) to [bend left=15] (3,3);
	\end{tikzpicture}
}
\newcommand{\folgerichtig}{
	\begin{tikzpicture}[yscale=.2,xscale=.2,overlay]
		\draw[line width=4,line join=bevel,blue] (0,1) to [bend left=30] (1,0) to [bend left=15] (3,3);
	\end{tikzpicture}
}
\newcommand{\falsch}{
	\begin{tikzpicture}[yscale=.2,xscale=.2,overlay]
		\draw[line width=4,line join=bevel,red] (0,2) to [bend left=5] (2,0) (0,0) to [bend left=15] (2,2);
	\end{tikzpicture}
}
\newcommand{\fsel}{
	\begin{tikzpicture}[yscale=.15,xscale=.15,overlay]
		\draw[line width=4,line join=bevel] (0,1) to [bend left=30] (1,0) to [bend left=15] (3,3);
		\draw[line width=3,line join=bevel,green] (0,1) to [bend left=30] (1,0) to [bend left=15] (3,3);
	\end{tikzpicture}
}
\newcommand{\felim}{
	\begin{tikzpicture}[yscale=.15,xscale=.15,overlay]
		\draw[line width=4,line join=bevel] (0,2) to [bend left=5] (2,0) (0,0) to [bend left=15] (2,2);
		\draw[line width=3,line join=bevel,red] (0,2) to [bend left=5] (2,0) (0,0) to [bend left=15] (2,2);
	\end{tikzpicture}
}

%\newtheorem{theorem}{Theorem}[section]
%\newtheorem{lemma}[theorem]{Lemma}
%\newtheorem{proposition}[theorem]{Proposition}
%\newtheorem{corollary}[theorem]{Corollary}
%
%\newenvironment{proof}[1][Proof]{\begin{trivlist}
%\item[\hskip \labelsep {\bfseries #1}]}{\end{trivlist}}
%\newenvironment{definition}[1][Definition]{\begin{trivlist}
%\item[\hskip \labelsep {\bfseries #1}]}{\end{trivlist}}
%\newenvironment{example}[1][Example]{\begin{trivlist}
%\item[\hskip \labelsep {\bfseries #1}]}{\hfill$\square$\end{trivlist}}
%%\newenvironment{example}[1][Example]{\itshape\begin{trivlist}
%%\item[\hskip \labelsep {\bfseries #1}]}{\hfill$\square$\end{trivlist}\normalfont}
%\newenvironment{remark}[1][Remark]{\begin{trivlist}
%\item[\hskip \labelsep {\bfseries #1}]}{\end{trivlist}}
%
%\newcommand{\qed}{\nobreak \ifvmode \relax \else
      %\ifdim\lastskip<1.5em \hskip-\lastskip
      %\hskip1.5em plus0em minus0.5em \fi \nobreak
      %\vrule height0.75em width0.5em depth0.25em\fi}
      
\newcommand{\picsheight}{60mm}
\newcommand{\paperreference}[3]{
	\vspace{2mm}
	\tikz{\node[draw=gray,line width=.5pt,fill=white,inner sep=0mm,blur shadow={shadow blur steps=5,shadow blur extra rounding=2pt}] {\href{#3}{\includegraphics[height=\picsheight]{papers/#1}}};\node[yshift=-15mm,rounded corners,fill=uulmaccent,blur shadow={shadow blur steps=5,shadow blur extra rounding=2pt}] {#2};}
}
\newcommand{\authorsays}[3]{
	\tikz{\node[draw=gray,line width=1pt,fill=white,inner sep=0mm,blur shadow={shadow blur steps=5,shadow blur extra rounding=2pt}] {\includegraphics[height=\picsheight]{authors/#1}};\node[yshift=-25mm,rounded corners,fill=white,blur shadow={shadow blur steps=5,shadow blur extra rounding=2pt}] {#2 says ``#3''};}
}

\usepackage{pgfplots}
	%\pgfplotsset{compat=1.9}
	\usepgfplotslibrary{units}
	\usepgfplotslibrary{fillbetween}
	\usepgfplotslibrary{patchplots}
\usepackage{pgfplotstable}
\usepackage{tikz}
	\usetikzlibrary{arrows,positioning,backgrounds,fit,mindmap,trees} 
	\usetikzlibrary{fadings,shapes.geometric}%spy,
	\usetikzlibrary{decorations,scopes,calc,decorations.pathreplacing,patterns,snakes}%
	\usetikzlibrary{shadows.blur} % used for shadows
	\tikzset{
		%Define standard arrow tip
		>=stealth',
		%Define style for boxes
		node1/.style={
		  rectangle,
		  rounded corners,
		  draw=black, very thick,
		  text width=9em,
		  minimum height=7em,
		  fill=background,
		  text centered,
		},
		node1s/.style={
		  rectangle,
		  rounded corners,
		  draw=black, very thick,
		  text width=7em,
		  minimum height=6em,
		  fill=background,
		  text centered,
		},
		node2/.style={
		  circle,
			dashed,
		  draw=black, very thick,
		  text width=6em,
		  minimum height=7em,
		  fill=background,
		  text centered,
		},
		node2s/.style={
		  circle,
			dashed,
		  draw=black, very thick,
		  text width=5em,
		  minimum height=6em,
		  fill=background,
		  text centered,
		},
		node3/.style={
		  rectangle,
		  rounded corners,
		  draw=black,
			inner xsep=0.5em,
			inner ysep=0.5em,
		},
		node4/.style={
		  rounded corners,
			draw,
			fill=orange!10,
		},
		% Define arrow style
		edge1/.style={
			->,
			thick,
			shorten <=2pt,
			shorten >=2pt,
		},
		edge2/.style={
			->,
			thick,
			dashed,
			shorten <=2pt,
			shorten >=2pt,
		},
		edge3/.style={
			<->,
			thick,
			shorten <=2pt,
			shorten >=2pt,
		},
		% Annotations in listings
		annotation/.style={
			fill opacity=0.15,
			right,
			yshift=0.25em,
			inner xsep=0.1em,
			inner ysep=0.5em,
		},
		redstyle/.style={fill=red5,postaction={pattern=dots,pattern color=white}},
		bluestyle/.style={fill=blue5,postaction={pattern=vertical lines,pattern color=white}},
		orangestyle/.style={fill=orange5,postaction={pattern=north east lines,pattern color=white}},
		greenstyle/.style={fill=green5,postaction={pattern=grid,pattern color=white}},
		graystyle/.style={fill=gray5,postaction={pattern=crosshatch,pattern color=white}},
	}
\newcommand{\tbox}[3]{
	\begin{pgfonlayer}{background}
		\filldraw [line width=4mm,join=round,blue!15]
		(#1.north -| #1.east) rectangle (#2.south -| #2.west);
		\node[fit=(#1)(#2)] {#3};
	\end{pgfonlayer}
}
\newcommand{\tpkt}[1]{\tikz[remember picture] \coordinate(#1);}
\newcommand{\annotation}[2]{
  \begin{tikzpicture}[remember picture,overlay]
		\foreach \x in {#1} {
			\node[annotation,fill=#2,fit=(\x 1)(\x 2)] {};
		}
  \end{tikzpicture}
}
\newcommand{\annnotation}[2]{
  \begin{tikzpicture}[remember picture,overlay]
		\foreach \x in {#1} {
			\node[annotation,fill=#2,fit=(\x 1)(\x 2)(\x 3)] {};
		}
  \end{tikzpicture}
}
\newcommand{\myunderline}[2]{
  \begin{tikzpicture}[remember picture,overlay]
		\foreach \x in {#1} {
			\node[yshift=.5mm] (x1) at (\x 1) {};
			\node[yshift=.5mm] (x2) at (\x 2) {};
			\draw[#2,thick,snake=snake,segment amplitude=.2mm,segment length=1mm] (x1.south) -- (x2.south);
		}
  \end{tikzpicture}
}
\ifprint{
	\newcommand{\errorunderline}[1]{\myunderline{#1}{black}}
	\newcommand{\warningunderline}[1]{\myunderline{#1}{black}}
	\newcommand{\infounderline}[1]{\myunderline{#1}{black}}
}{
	\newcommand{\errorunderline}[1]{\myunderline{#1}{red}}
	\newcommand{\warningunderline}[1]{\myunderline{#1}{orange}}
	\newcommand{\infounderline}[1]{\myunderline{#1}{blue}}
}
\newcommand{\reference}[3]{
	\begin{tikzpicture}[overlay]
		\foreach \x in {#1} {
			\path[shorten <= 2pt,shorten >= 2pt,->,line width=1pt,#2] (\x 1) edge #3 (\x 2);
		}
	\end{tikzpicture}
}
	
% Tortendiagramme
\newcommand{\slice}[4]{
  \pgfmathparse{0.5*#1+0.5*#2}
  \let\midangle\pgfmathresult

  % slice
  \draw[thick,
	%fill=background
	] (0,0) -- (#1:1) arc (#1:#2:1) -- cycle;

  % outer label
  \node[label=\midangle:#4] at (\midangle:1) {};

  % inner label
  \pgfmathparse{min((#2-#1-10)/110*(-0.3),0)}
  \let\temp\pgfmathresult
  \pgfmathparse{max(\temp,-0.5) + 0.8}
  \let\innerpos\pgfmathresult
  \node at (\midangle:\innerpos) {#3};
}
\newcounter{a}
\newcounter{b}
\newcommand{\mypiechart}[2]{
	\begin{tikzpicture}[scale=#1]
		\setcounter{a}{0}
		\setcounter{b}{0}
		\foreach \p/\t in {#2}
			{
				\setcounter{a}{\value{b}}
				\addtocounter{b}{\p}
				\slice{\thea/100*360}
							{\theb/100*360}
							{}{\p~\% \t}
			}
	\end{tikzpicture}
}

\newcommand{\mglass}[1]{
	\scalebox{#1}{
		\begin{tikzpicture}[overlay]
			\node[circle,draw=gray,line width=1,fill=blue,fill opacity=.2,inner sep=2mm] {} edge[draw=gray,line width=1,double distance=1pt] (.33,-.33);
		\end{tikzpicture}
	}
	%\begin{tikzpicture}[scale=#1,overlay]
		%\node[scale=#1,circle,draw=gray,line width=#1,fill=blue,fill opacity=.2,inner sep=2mm] {} edge[scale=#1,draw=gray,line width=#1,double distance=#1pt] (.4,-.4);
	%\end{tikzpicture}
}


% Based on the colors of TU Braunschweig

\definecolor{blue}{named}{tuBlue}
\definecolor{orange}{named}{tuOrange}
\definecolor{green}{named}{tuGreen}
\definecolor{red}{named}{tuRed}

\colorlet{blue1}{blue}
\colorlet{blue2}{blue!75!black}
\colorlet{blue3}{blue!50!black}
\colorlet{blue4}{blue!50!white}
\colorlet{blue5}{blue!25!white}

\colorlet{orange1}{orange}
\colorlet{orange2}{orange!75!black}
\colorlet{orange3}{orange!50!black}
\colorlet{orange4}{orange!50!white}
\colorlet{orange5}{orange!25!white}

\colorlet{green1}{green}
\colorlet{green2}{green!75!black}
\colorlet{green3}{green!50!black}
\colorlet{green4}{green!50!white}
\colorlet{green5}{green!25!white}

\colorlet{red1}{red}
\colorlet{red2}{red!75!black}
\colorlet{red3}{red!50!black}
\colorlet{red4}{red!50!white}
\colorlet{red5}{red!25!white}

\colorlet{gray1}{gray}
\colorlet{gray2}{gray!75!black}
\colorlet{gray3}{gray!50!black}
\colorlet{gray4}{gray!50!white}
\colorlet{gray5}{gray!25!white}

\definecolor{background}{named}{white}
\definecolor{bgborder}{named}{black}
\definecolor{comment}{named}{red3}

%\definecolor{coqred}{named}{red1}
%\definecolor{coqblue}{named}{blue1}
\definecolor{coqcomment}{named}{comment}
\definecolor{coqred}{named}{black}
\definecolor{coqblue}{named}{black}
%\definecolor{coqcomment}{named}{black}

\definecolor{acolor1}{named}{red}
\definecolor{acolor2}{named}{blue}
\definecolor{gacolor1}{gray}{0.34}%86
\definecolor{gacolor2}{gray}{0.37}%95
\definecolor{highlight}{gray}{0.9}

\definecolor{pdflinkcolor}{named}{blue3}
\definecolor{pdfcitecolor}{named}{green3}

\definecolor{grayed}{gray}{0.66}

\usepackage{listings} % source code listings

%\newcommand{\modsep}{\vspace{1mm}\hrule\vspace{1mm}\hrule\vspace{1mm}}
\newcommand{\modsep}{\hrule\vspace{1mm}\hrule}
\newcommand{\featuremodule}[1]{\hfill{\rmfamily feature module \feature{#1}}}
\newcommand{\featureinterface}[1]{\hfill{\rmfamily interface of \feature{#1}}}
\newcommand{\configuration}[1]{\hfill{\rmfamily\{\feature{#1}\}}}
\newcommand{\product}[1]{\hfill{\rmfamily\{\feature{#1}\}}}
%\newcommand{\comment}[1]{\hfill{\rmfamily#1}}
%\newcommand\fheader[1]{\hfill{\footnotesize Feature module \em #1~~~\vskip -1ex}}
%\newcommand\pheader[1]{\hfill{\footnotesize Product \em \{#1\}~~~\vskip -1ex}}
%\newcommand\iheader[1]{\hfill{\footnotesize Interface of \em #1~~~\vskip -1ex}}
%\newcommand\sheader[1]{\hfill{\footnotesize Specification of \em #1~~~}}

% JML keywords in comments
\ifprint{
	\newcommand{\jmlkeyword}[1]{\color{darkgray}\textbf{#1}}
}{
	\newcommand{\jmlkeyword}[1]{\color{green3}\textbf{#1}}
}
\newcommand{\invariant}{\jmlkeyword{invariant}}
\newcommand{\public}{\jmlkeyword{public}}
\newcommand{\static}{\jmlkeyword{static}}
\newcommand{\normalbehavior}{\jmlkeyword{normal\_behavior}}
\newcommand{\exceptionalbehavior}{\jmlkeyword{exceptional\_behavior}}
\newcommand{\pure}{\jmlkeyword{pure}}
\newcommand{\helper}{\jmlkeyword{helper}}
\newcommand{\requires}{\jmlkeyword{requires}}
\newcommand{\assignable}{\jmlkeyword{assignable}}
\newcommand{\ensures}{\jmlkeyword{ensures}}
\newcommand{\also}{\jmlkeyword{also}}
\newcommand{\nothing}{\jmlkeywordb{nothing}}
\newcommand{\assert}{\jmlkeyword{assert}}
\newcommand{\intjml}{\jmlkeyword{int}}
\newcommand{\mytrue}{\jmlkeyword{true}}
\newcommand{\myfalse}{\jmlkeyword{false}}
\newcommand{\nulljml}{\jmlkeyword{null}}
\newcommand{\this}{\jmlkeyword{this}}
\newcommand{\instanceof}{\jmlkeyword{instanceof}}

\ifprint{
	\newcommand{\jmlkeywordb}[1]{\color{darkgray}\textbackslash\textbf{#1}}
}{
	\newcommand{\jmlkeywordb}[1]{\color{green3}\textbackslash\textbf{#1}}
}
\newcommand{\existsjml}{\jmlkeywordb{exists}}
\newcommand{\foralljml}{\jmlkeywordb{forall}}
\newcommand{\old}{\jmlkeywordb{old}}
\newcommand{\result}{\jmlkeywordb{result}}
\newcommand{\original}{\jmlkeywordb{original}}

\newcommand{\finalmethod}{\jmlkeywordb{final\_method}}
\newcommand{\finalcontract}{\jmlkeywordb{final\_contract}}
\newcommand{\conjunctive}{\jmlkeywordb{conjunctive\_contract}}
\newcommand{\cumulative}{\jmlkeywordb{cumulative\_contract}}
\newcommand{\consecutive}{\jmlkeywordb{consecutive\_contract}}

%\renewcommand\lstlistingname{Quelltext}

\lstdefinestyle{java}{
%code formatting
	language=Java,
	tabsize=4,
	breaklines=false,
	basicstyle=\sf\small\selectfont,
	commentstyle=\fontshape{it}\color{darkgray}\selectfont,
	keywordstyle=\fontseries{b}\selectfont,%\color{orange3}
	stringstyle=\selectfont,
	columns=fullflexible, 
	%basicstyle=\fontfamily{pcr}\footnotesize\selectfont,
	%commentstyle=\fontshape{it}\color{darkgray}\selectfont,
	%keywordstyle=\fontseries{b}\selectfont,
	%stringstyle=\fontfamily{pcr}\selectfont,
%line numbering
	numbers=none,%left, right, none
	numberstyle=\footnotesize,
%frame properties
	captionpos=b,
	frame=single,%trblTRBL
	framesep=3pt,
	xleftmargin=4pt,
	xrightmargin=4pt,
	rulecolor=\color{bgborder},
	escapechar=|,
	firstnumber=auto,
	showstringspaces=false,
}

\lstdefinestyle{xml}{
	style=java,
	tabsize=2,
	language=xml,
}

\lstdefinestyle{featureidexml}{
	style=xml,
	morekeywords={featureModel, struct, feature, name, abstract, mandatory, and, alt, or, constraints, rule, imp, var, comments},
}

\lstdefinestyle{jak}{
	style=java,
	morekeywords={layer, refines, Super(), Super},
}

\lstdefinestyle{featurehouse}{
	style=java,
	morekeywords={original, refines},
}

\lstdefinestyle{deltaj}{
	style=java,
	morekeywords={original, core, delta, when, after, before, modifies, adds, removes},
}

\lstdefinestyle{aspectj}{
	style=java,
	morekeywords={aspect, after, call},
}

\lstdefinestyle{ifdef}{
  morecomment=[l]{\#},
}

\lstdefinestyle{interface}{
	morekeywords={requires, provides},
}

\ifprint{
	\lstdefinestyle{jml}{
		commentstyle=\color{darkgray}\selectfont,%\fontshape{it}
	}
		%moredelim=*[s][\itshape]{/*}{*/},
		%emph={@},
		%emphstyle={\normalfont\rmfamily\selectfont},
}{
	\lstdefinestyle{jml}{
		commentstyle=\color{green3}\selectfont,%\fontshape{it}
	}
}

\lstdefinestyle{small}{
	basicstyle=\sf\small\selectfont
}
\lstdefinestyle{footnotesize}{
	basicstyle=\sf\footnotesize\selectfont
}
\lstdefinestyle{scriptsize}{
	basicstyle=\sf\scriptsize\selectfont
}
\lstdefinestyle{tiny}{
	basicstyle=\sf\tiny\selectfont
}

%%%%% Coq %%%%%%%%%%%%%%%%%%%%%%%%%%%%%%%%%%%%%%%%%%%%%%%%%%%%%%%%%%%%%%%%%%%%%

\lstdefinelanguage{Coq}{
  morekeywords={Variable,Inductive,CoInductive,Fixpoint,CoFixpoint,%
      Definition,Lemma,Theorem,Axiom,Local,Save,Grammar,Syntax,Intro,%
      Trivial,Qed,Intros,Symmetry,Simpl,Rewrite,Apply,Elim,Assumption,%
      Left,Cut,Auto,Unfold,Exact,Right,Hypothesis,Pattern,Destruct,%
      Constructor,Defined,Fix,Record,Proof,Induction,Hints,Exists,let,in,%
      Parameter,Split,Red,Reflexivity,Transitivity,if,then,else,Opaque,%
      Transparent,Inversion,Absurd,Generalize,Mutual,Cases,of,end,Analyze,%
      AutoRewrite,Functional,Scheme,params,Refine,using,Discriminate,Try,%
      Require,Load,Import,Scope,Set,Open,Section,End,match,with,Ltac,%
			exists,forall
			%Case
	},% 
  sensitive, %
  morecomment=[n]{(*}{*)},%
  morestring=[d]",%
  literate=
   {=>}{{$\Rightarrow$}}1
	 {>->}{{$\rightarrowtail$}}2
	 {<->}{{$\leftrightarrow$}}2
	 {->}{{$\to$}}1
    {\/\\}{{$\wedge$}}1
    {|-}{{$\vdash$}}1
    {\\\/}{{$\vee$}}1
    {~}{{$\sim$}}1
    %{forall}{{$\forall$}}1 problem if your names contain forall or exists
    %{exists}{{$\exists$}}1
    %{<>}{{$\neq$}}1 indeed... no.
}[keywords,comments,strings]%

\lstdefinestyle{coqblack}{
%code formatting
	language=Coq,
	tabsize=4,
	breaklines=false,
	basicstyle=\sf\small\selectfont,
	commentstyle=\fontshape{it}\selectfont,
	keywordstyle=\fontseries{b}\selectfont,%\color{orange3}
	stringstyle=\selectfont,
	columns=fullflexible, 
	%columns=[c]fixed,
	%flexiblecolumns=false,
	%basicstyle=\fontfamily{pcr}\footnotesize\selectfont,
	%commentstyle=\color{coqcomment}\fontshape{it}\selectfont,
	%keywordstyle=\color{coqblue}\fontseries{b}\selectfont,
	%stringstyle=\fontfamily{cmr}\selectfont,
%line numbering
	numbers=none,
	numberstyle=\footnotesize,
%frame properties
	captionpos=b,
	frame=single,%trblTRBL
%	framesep=3pt,
	xleftmargin=4pt,
	xrightmargin=4pt,
%  aboveskip=4ex,
	rulecolor=\color{bgborder},
	%backgroundcolor=\color{background},
%forgotten keywords
	morekeywords={Module, Fact, Export, Tactic, Notation, Prop, Combined, Variables},
%emphasizing
  emph={Variable, Hypothesis, Module, Fact, Lemma, Theorem, Ltac, Inductive, Fixpoint, Definition, Parameter, with, admit, Variables},
	emphstyle={\fontseries{b}\selectfont},
  escapeinside={ß}{ß} % escapeinside={$}{$}
}

\ifprint{
	\lstdefinestyle{coq}{
		style=coqblack
	}
}{
	\lstdefinestyle{coq}{
		style=coqblack,
		commentstyle=\fontshape{it}\color{darkgray}\selectfont,
		keywordstyle=\fontseries{b}\color{blue3}\selectfont,%\color{orange3}
		emphstyle={\color{red3}\fontseries{b}\selectfont},
	}
}

\lstdefinestyle{slides}{
	backgroundcolor=\color{white}
}

\usepackage{fancyvrb}
\newenvironment{freemove}{\begin{picture}(320,190)}{\end{picture}}
\newcommand{\freetext}[4]{\put(#1,#2){\begin{minipage}[t]{#3\linewidth}{\large #4}\end{minipage}}}

\tikzset{onslide/.code args={<#1>#2}{%
		\only<#1>{\pgfkeysalso{#2}}
}}
\makeatletter
\newenvironment{savelst}{\VerbatimEnvironment\begin{VerbatimOut}{vsave.tmp}}
	{\end{VerbatimOut}}

\newcommand{\savedlst}{\lstinputlisting{vsave.tmp}}

\newcommand{\savedlstbox}{\settowidth\@tempdima{%
		{\renewcommand{\pause}{}  \savedlst}}
	\minipage{\@tempdima}\savedlst\endminipage}

\newenvironment{lstbox}
{\VerbatimEnvironment\begin{savelst}}
	{\end{savelst}\savedlstbox}    
\makeatother

\lstdefinestyle{customc}{
	%  belowcaptionskip=1\baselineskip,
	%  breaklines=true,
	%  frame=L,
	%  xleftmargin=\parindent,
	language=C,
	%  showstringspaces=false,
	frame=single,
	framerule=0.9pt,
	framesep=9pt,
	basicstyle=\ttfamily\bfseries,
	%  keywordstyle=\color{purple!40!black},
	%  keywordstyle=\color{blue},
	commentstyle=\color{green!40!black},
	identifierstyle=\color{red},
	directivestyle=\color{gray},
	stringstyle=\color{orange},
	escapechar=|,
	tabsize=4,
}

\lstdefinestyle{customjava}{
	%  belowcaptionskip=1\baselineskip,
	%  breaklines=true,
	%  frame=L,
	%  xleftmargin=\parindent,
	language=Java,
	%  showstringspaces=false,
	frame=single,
	framerule=0.9pt,
	framesep=9pt,
	basicstyle=\ttfamily\bfseries,
	%  keywordstyle=\color{purple!40!black},
	keywordstyle=\color{blue},
	commentstyle=\color{green!40!black},
	identifierstyle=\color{black},
	%  directivestyle=\color{gray},
	stringstyle=\color{orange},
	escapechar=|,
	tabsize=4,
}


\newcommand{\partslide}[1]{
	\begin{frame}[plain]
	\centering\vspace{10mm}
	\partpage\vspace{5mm}
	\includegraphics[height=0.6\textheight]{../../../pics/emotions/#1}
\end{frame}
}

\newcommand{\ttquote}[3]{
{\color{blue}\hfill\fbox{#1}\\[2mm]}
\includegraphics[width=.7\linewidth]{../../../2017/fmi-SE/pics/#2}
\\~\\
\fbox{\includegraphics[width=.8\linewidth]{../../../2017/fmi-SE/pics/#2-q#3}}
}

\newcommand{\paperreference}[2]{
\tikz{\node[draw=lightgray,fill=white,inner sep=5mm,blur shadow={shadow blur steps=5,shadow blur extra rounding=2pt}] {\includegraphics[width=.33\textwidth]{#1}};\node[rounded corners,fill=white] {\footnotesize #2};}
}

\newcommand{\partframe}[3]{
\frame[plain,label=#1]{
	\begin{center}
		\Large\hyperlink{parts}{#2}
	\end{center}
	\vfill
	
	#3
}
}

\newcommand{\centernote}[1]{
	\begin{tikzpicture}[remember picture,overlay,align=left]
	\node[preaction={fill=white},preaction={fill=red!25!white,path fading=north,fading angle=15},draw,thick,rounded corners,inner sep=10pt,yshift=3mm] at (current page.center) {\large #1};
	\end{tikzpicture}
}
